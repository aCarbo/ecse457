\documentclass[12pt]{article}

\usepackage[margin=1in]{geometry}
\setlength{\parindent}{0pt}

\begin{document}
\title{Progress Report 2}
\date{}
\maketitle
\section*{Project Title} 
Transmission Control System Design using CRIO Real Time Controller:
Data Gathering and Actuator Control
\section*{Team Members}
Group 35 \newline

\hangindent=17.62482pt
\hangafter=1
Shayan Ahmad \\
260350431 \\
shayan.ahmad@mail.mcgill.ca \newline

\hangindent=17.62482pt
\hangafter=1
Alejandro Carboni Jimenez \\
260523638 \\
alejandro.carbonijimenez@mail.mcgill.ca \newline

\hangindent=17.62482pt
\hangafter=1
Aditya Saha \\
260453165 \\
aditya.saha@mail.mcgill.ca

\section*{Supervisors}
\hangindent=17.62482pt
\hangafter=1
Yingxuan Duan PhD.\\ 
Research Associate\newline 

\hangindent=17.62482pt
\hangafter=1
Benoit Boulet PhD.\\
Associate Professor - Department of Electrical and Computer Engineering\\
Associate Chair - Operations

\newpage
\section*{Group Meetings with Advisor}
A preliminary group meeting was held on January 12th to establish the
strategy for this semester's work. \newline

A meeting was held with our immediate supervisor on January 13th. During the
meeting, short-term goals and the meeting schedule for the rest of the semester
was discussed. During the meeting, we also received feedback on our final 
deliveravbles from last semester. Following this meeting, follow-up meeting was
held to check our progress and verify the feasability of the new goals for 
this term.\newline

A last meeting was held with our supervisor on January 19th to demo the work
accomplished during the first work session. 

\section*{Project Readings}
Specification document corresponding to the newly acquired data logging module.

\section*{Recent Progress}
The J1939 variant of the CAN communication protocol was established. An Arduino
was set up as a write node, and both an Arduino and a CAN handle connected to a
laptop were set up as read nodes. Rigorous testings were carried out 
subsequently with both two and three node configuration to check for integrity 
of the transmitted data, bus arbitration and collision handling. Furthermore a
sensor was introduced to the CAN network to to simulate a more realistic model
and tests checking for consistency of the network were carried out.\newline

The project source code base has been updated and shared with the immediate 
supervisor for ease of reference and discussion. 

\section*{Future Plans}
In the following weeks the system implementation will be updated to include 
the data logging module. Furthermore new force sensor will be acquired and 
calibration tests will be exercised. A driving circuit will also be designed
to synergize with the Arduino and close the feedback loop. Once the supervisor
provides the implementation of CAN transmission on cRIO module, the individual
system components will be integrated and tested.

\section*{Group Work Report}
All the work so far has been evenly shared between two team members, Aditya and
Alejandro. The third team member has officially informed of his intention to 
pursue an internship in a different city and work remotely. However due to the 
nature of current work necessitating lab attendance, no work has been delegated 
to him. At the moment of writing this report, Shayan Ahmad has not made any
contributions to the project.

\end{document}
