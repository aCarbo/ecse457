\documentclass[12pt]{article}

\usepackage[margin=1in]{geometry}
\setlength{\parindent}{0pt}

\begin{document}
\title{Progress Report 2}
\date{}
\maketitle
\section*{Project Title} 
Transmission Control System Design using CRIO Real Time Controller:
Data Gathering and Actuator Control
\section*{Team Members}
Group 35 \newline

\hangindent=17.62482pt
\hangafter=1
Shayan Ahmad \\
260350431 \\
shayan.ahmad@mail.mcgill.ca \newline

\hangindent=17.62482pt
\hangafter=1
Alejandro Carboni Jimenez \\
260523638 \\
alejandro.carbonijimenez@mail.mcgill.ca \newline

\hangindent=17.62482pt
\hangafter=1
Aditya Saha \\
260453165 \\
aditya.saha@mail.mcgill.ca

\section*{Supervisors}
\hangindent=17.62482pt
\hangafter=1
Yingxuan Duan PhD.\\ 
Research Associate\newline 

\hangindent=17.62482pt
\hangafter=1
Benoit Boulet PhD.\\
Associate Professor - Department of Electrical and Computer Engineering\\
Associate Chair - Operations

\newpage
\section*{Group Meetings with Advisor}
Since the previous progress report, meeting with the supervisor has primarily
served to demonstrate the work accomplished so far. Prioritizing the next steps
of the project was done together as well. 


\section*{Project Readings}
Specification document corresponding to the newly acquired data logging module.

\section*{Recent Progress}
The J1939 variant of the CAN communication protocol was established. An Arduino
was set up as a write node, and both an Arduino and a CAN handle connected to a
laptop were set up as read nodes. Rigorous testings were carried out 
subsequently with both two and three node configuration to check for integrity 
of the transmitted data, bus arbitration and collision handling. Furthermore a
sensor was introduced to the CAN network to to simulate a more realistic model
and tests checking for consistency of the network were carried out.\newline

The project source code base has been updated and shared with the immediate 
supervisor for ease of reference and discussion. The final deliverables will
include the proper documentation and presentation of this code base for future
use in the lab.

\section*{Future Plans}
In the following weeks the system implementation will be updated to include 
the data logging module. Furthermore, a new force sensor will be acquired and 
calibration tests will be exercised. An analog to digital converter will be 
designed for this new sensor, and implemented for use with the Arduino.
A driving circuit will also be designed to synergize with the Arduino and close
the feedback loop. Once the supervisor verifies the existing CAN transmission 
capabilities of the cRIO module, the individual system components will be 
integrated and tested.

\section*{Group Work Report}
All the work so far has been evenly shared between two team members, Aditya and
Alejandro. The third team member has officially informed of his intention to 
pursue an internship in a different city and work remotely. However due to the 
nature of current work necessitating lab attendance, he will coordinate directly
with the supervisor to determine what tasks he can be assigned. The result of
their coordination should be available for the next progress report and will
be discussed then.

\end{document}
